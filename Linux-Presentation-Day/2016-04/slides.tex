\documentclass[t]{beamer}
\usepackage[ngerman]{babel}
\usepackage{textcomp}
\usepackage{subfigure}
\usepackage{graphicx}
\usepackage{fancyhdr}
\usepackage{tgpagella}
\usepackage{textpos}
\usepackage{hyperref}
\usepackage{tcolorbox}

\usepackage{fontspec}
\setmainfont{Titillium}
% \setsansfont{}

\usetheme{Dresden}
\useoutertheme{infolines}

\setbeamerfont*{quote}{family=\sffamily}
\addtobeamertemplate{quote begin}{}{\begin{tcolorbox}}
\addtobeamertemplate{quote end}{\end{tcolorbox}}{}

\title{Linux Presentation Day April 2016}
\author{\texttt{https://fsfw-dresden.de/lpd}}
\date[30.04.2016]{30. April 2016}

% TODO: fsfw Farbschema
% LPD Logo in die Ecke
\addtobeamertemplate{frametitle}{}{%
    \begin{textblock*}{130mm}(.82\textwidth,-0.5cm)
        \includegraphics[width=2.25cm]{pdf-gen/fsfw-logo.pdf}
\end{textblock*}}

\hyphenation{
  Regel
}

\AtBeginSection[] % Do nothing for \section*
{
  \begin{frame}<beamer>
    \frametitle{Gliederung}
    \tableofcontents[currentsection]
  \end{frame}
}
\begin{document}
\begin{frame}
  \begin{center}%
% LPD Logo
\includegraphics[width=4cm]{pdf-gen/fsfw-logo-with-text.pdf}\\%
\vspace*{-1em}{Freie Software Freies Wissen}\\[1em]
\structure{\Large Linux Presentation Day 2016}
  \end{center}
\end{frame}

\begin{frame}
  \frametitle{Gliederung}
  \tableofcontents
\end{frame}

\section{Einführung und Vorstellung}
\begin{frame}
  \frametitle{Wer sind wir?}
  \begin{itemize}
  \item FSFW – Hochschulgruppe für Freies Wissen und Freie Software
    \begin{itemize}
    \item Hochschulgruppe an der TU Dresden
    \item Mitglieder von der HTW und Mitglieder ohne
      Hochschulaffiliation
    \end{itemize}
  \item LUG DD – Linux-User-Group Dresden
  \item c3d2 – Chaos Computer Club Dresden
  \item Datenkollektiv
  \item StuRa der HTW (Räume und Budget)
  \item IT4R – IT for refugees
  \end{itemize}
  % Pro Initiative eine Slide (und eine Übersichtsslide)
  % URLs
\end{frame}

\begin{frame}
  \frametitle{Warum machen wir das?}
  \begin{itemize}
  \item weil heute der internationale Linux-Presentation-Day ist
  \item aktuelles Interesse an Linux-Systemen (Windows 10, Steam
    für Linux, Edward Snowden)
  \item wir verwenden es selbst gerne
  \end{itemize}
\end{frame}

\section{Über Linux und Freie Software}
\begin{frame}
  \frametitle{Was ist Linux?}
  \begin{quote}
    Als Linux oder GNU/Linux […] bezeichnet man in
    der Regel freie, unix-ähnliche Mehrbenutzer-Betriebssysteme, die
    auf dem Linux-Kernel und wesentlich auf GNU-Software
    basieren.\footnote{\url{https://de.wikipedia.org/wiki/Linux}}
  \end{quote}
  \begin{center}
    \Large … das bedarf Erläuterung.
  \end{center}
\end{frame}

\begin{frame}
  \frametitle{Was ist ein Betriebssystem?}
  \begin{block}{Kernel}
    macht die Drecksarbeit – stellt die Umgebung für Programme bereit
    \begin{itemize}
    \item Scheduling, Speicherverwaltung, Hardwaretreiber, Interrupts,
      DMA, Netzwerk, …
    \end{itemize}
  \end{block}
  \begin{block}{Userland}
    \begin{itemize}
    \item Graphische Oberfläche
    \item Standardwerkzeuge
    \item Standardbibliotheken
    \end{itemize}
  \end{block}
\end{frame}

\begin{frame}
  \frametitle{Was bedeutet „Freie Software“?}
  \begin{block}{Grundfreiheiten für Software}
    \begin{enumerate}
    \item Zu jedem Zwecke ausführen
    \item Untersuchen und Verändern
    \item Weitergeben
    \item Weitergeben von Änderungen
    \end{enumerate}
  \end{block}
  % Entwicklungsmodell – OpenSource
\end{frame}

\begin{frame}
  \frametitle{Was ist Linux? – for real}
  \begin{itemize}
  \item bezeichnet eigentlich den Linux-Kernel
  \item im allgemeinen Sprachgebrauch: ganzes Betriebssystem
  \item Freie Software – größtenteils
  \item große Vielfalt an Ausprägungen (Server, Embedded, diverse
    Desktop-Systeme)
  \end{itemize}
\end{frame}

% Geschichte von Linux Slide

\section{Begriffe}
\begin{frame}
  \frametitle{Einige Begriffe}
  \begin{block}{Distribution}
    Sammlung von Software und Standardkonfiguration, Populäre
    Distributionen: Fedora, Debian, Mint, Ubuntu, Arch Linux
    % Symbolbild: Stammbaum der Distros
  \end{block}
  \begin{block}{Desktop-Umgebung}
    Graphische Oberfläche und Standardanwendungen, Populäre
    Desktop-Umgebungen: Unity (Ubuntu), Gnome, Mate, Cinnamon, KDE
  \end{block}
  \begin{block}{Datei-Manager}
    Entspricht dem Windows-Explorer oder dem Mac OS Finder – Jede
    Desktop-Umgebung hat einen eigenen.
  \end{block}
\end{frame}

\section{Gründe für und gegen Linux}
\begin{frame}
  \frametitle{Warum Freie Software?}
  \begin{itemize}
  \item Vier Freiheiten
    % wichtige anmerkung: => Verifizierbarkeit
  \item keine Lizenzgebühren
  \item Unterstützung älterer Hardware
  \item ethische Argumente
  \item Zugänglichkeit für Bildung
  \item Nachhaltigkeit
  \item Sicherheit${}^\text{[citation needed]}$
  \end{itemize}
\end{frame}

\begin{frame}
  \frametitle{Kontrolle über Updates}
  \begin{itemize}
  \item Updates werden installiert, wenn und wann der Nutzer das möchte
  \item kein verlängertes Herunterfahren durch Updates
  \item Sicherheitsupdates können gesondert installiert werden
  \end{itemize}
\end{frame}

\begin{frame}
  \frametitle{Konfigurierbarkeit und Anpassbarkeit}
  \begin{itemize}
  \item große Auswahl aus heterogenem Ökosystem
  \item für jeden etwas dabei (alte, schwächere Hardware; minimale
    Nerd-Systeme; ...)
  \item die meiste Software ist sehr gut konfigurierbar
  \end{itemize}
\end{frame}

\begin{frame}
  \frametitle{Privacy}
  \begin{itemize}
  \item telefoniert nicht nach Hause
  \item Festplattenverschlüsselung ist typischerweise einfach (ein
    Klick bei der Installation und ein Passwort angeben)
  \end{itemize}
\end{frame}

\begin{frame}
  \frametitle{Sicherheit}
  \begin{itemize}
  \item Vorteil durch kleineren Marktanteil im Desktopbereich
  \item Vorteil durch großen Marktanteil auf Serversystemen
  \item Community reagiert typischerweise schnell auf
    Sicherheitslücken
    % mündlich: ob man auch Updates bekommt ...
  \end{itemize}
\end{frame}

\begin{frame}
  \frametitle{Paketverwaltung}
  \begin{itemize}
  \item Programme komfortabel und sicher installieren
  \item von der Distribution bereitgestellt
  \item für viele \emph{das} Killerfeature
  \item App-Store auf Steroiden – seit 1998
    % mündlich: Abhängigkeiten, Deduplizierung, Katzenbilder
    % Programme für $alles direkt verfügbar ohne windige
    % Download-Websites
  \end{itemize}
\end{frame}

\begin{frame}
  \frametitle{Nutzerfreundlichkeit}
  \begin{itemize}
  \item schlechter Ruf nicht mehr gerechtfertigt
  \item mit modernen Desktopumgebungen mindestens so komfortabel wie
    Windows
  \item keine Angst vor der Konsole
    \begin{itemize}
    \item Kenntnis nicht zwingend notwendig
    \item macht das Leben für erfahrene Benutzer aber viel einfacher
    \end{itemize}
  % mündlich: it's a feature, not a bug
  % Symbolbolbild: Terminal
  % Eventuell eigene Slide für die Konsole
  \end{itemize}
\end{frame}

\begin{frame}
  \frametitle{Probleme mit Linux}
  \begin{itemize}
  \item exotische Hardware (NFC, seltener: WLAN, sehr neue Graphikkarten)
  \item Graphikleistung oft schlechter ($\rightarrow$ Spiele)
  \item Verfügbarkeit von spezieller Software (z.\,B. CAD)
  \item Accessibility
    % mündlich: NFC erläutern
  \end{itemize}
\end{frame}

\begin{frame}
  \frametitle{Linux neben Windows}
  \begin{itemize}
  \item im Prinzip kein Problem
  \item man bekommt dann ein Bootmenü zum Auswählen
  \item einziges praktisches Problem: Windows zerlegt den Bootloader
    bei größeren Updates (das muss man dann manuell beheben)
  \item viele Windows-Programme laufen auch unter wine (ältere Spiele,
    Bürosoftware, ...)
  \item falls Leistung oder direkter Hardwarezugriff nicht wichtig
    sind: Windows in einer VM
  \end{itemize}
  % TODO: Screenshot von grubd
  % zwei Folien: Dual Boot/Alternativen zu Dual Boot
  % zweite Slide: VM und wine erklären
  % Programmnamen Mark-Upen
\end{frame}

\section{Programm}
\begin{frame}
  \frametitle{Was gibt es jetzt noch?}
  \begin{itemize}
  \item Linux-Install-Party
  \item Computer mit Linux zum Ausprobieren
  \item Vorstellung der beteiligten Initiativen
  \item Mate, Wostok und Kekse
  \end{itemize}
\end{frame}

\section{Veranstaltungshinweise}
\begin{frame}
  \frametitle{Linux Helpdesk}
  \begin{itemize}
  \item LUG
  \item fsfw
  \end{itemize}
\end{frame}

\begin{frame}
  \frametitle{Interesse mitzumachen?}
  \begin{description}
  \item[Web] \url{https://fsfw-dresden.de}\\
    \begin{center}
      \hspace*{-6em}
      \includegraphics[width=4cm]{pdf-gen/website-qr.pdf}
    \end{center}
  \item[Mailingliste] \url{http://lists.fsfw-dresden.de/mailman/listinfo/discuss} (auch oben verlinkt)
  \item[Treffen] SLUB (Hauptgebäude), Raum -2.115, Donnerstags in ungeraden Wochen, 18:30 Uhr
  \end{description}
\end{frame}

\end{document}
