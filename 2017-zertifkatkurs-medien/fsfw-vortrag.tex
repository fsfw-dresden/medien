\documentclass{beamer}

\usepackage{fontspec}
\usepackage[ngerman]{babel}

\usetheme{FSFW}

\begin{document}

\title{Digitale Selbstverteidigung}
\subtitle{Freie Software und Freie Inhalt für die Schule}

\maketitle

\begin{frame}
  \frametitle{Beispiel: LibreOffice}

  % - einige haben weniger Geld und können sich nicht die aktuelle Technik
  % leisten
  % - offene Standards

\end{frame}

\begin{frame}
  \frametitle{Beispiel: NextCloud}

  % - Mobbing, Tracking, juristischer Datenschutz
  % - Nextcloud, Anbieten von eigene Infrastruktur

\end{frame}

\begin{frame}
  \frametitle{Beispiel: Gimp und Inkscape}

\end{frame}

\begin{frame}
  \frametitle{Weitere Beispiele}

  GeoGebra, Stellarium, Minetest, Wiki,

\end{frame}

\begin{frame}
  \frametitle{Software-Freiheiten}

  % TODO: Noch mehr auf Inhalte von
  % https://www.gnu.org/philosophy/free-sw.en.html
  % eingehen

  \onslide<+->

  \begin{enumerate}

  \item<+-> Software für jeden Anforderung benutzen zu können
    \begin{itemize}
    \item Erlaubt die uneingeschränkte Verwendung von Programmen
    \item Schützt vor komplexen Nutzungsvereinbarungen und versehentlichen
      Lizenzverstößen; Verwendung wird erheblich vereinfacht.
    \end{itemize}

  \item<+-> Software untersuchen und modifizieren zu können
    \begin{itemize}
    \item Erlaubt das Verständnis vorhandener Programme, die Anpassung an
      spezielle Anforderungen, und die Überprüfung der Korrektheit
    \item Schützt vor Abhängigkeiten und unerwünschtem Verhalten
    \end{itemize}

  \item<+-> Kopien der Software weitergeben zu können
    \begin{itemize}
    \item Erlaubt allen die Verwendung gleicher Programme, ohne Berücksichtigung
      von sozialem oder finanziellem Status, und fördert damit Zusammenarbeit
    \item Schützt vor sozialer Ausgrenzung
    \end{itemize}

  \item<+-> Angepasste Kopien weitergeben zu können
    \begin{itemize}
    \item Erlaubt allen Beteiligten, von Anpassungen anderer zu profitieren
    \item Schützt vor Abhängigkeiten bei Fehlerkorrekturen und vor unerwünschtem
      Verhalten der Software
    \end{itemize}
  \end{enumerate}

\end{frame}

\end{document}

% Local Variables:
% TeX-engine: luatex
% End:

%  LocalWords:  Nutzungsvereinbarungen
