\documentclass{beamer}

\usepackage{fontspec}
\usepackage[ngerman]{babel}
\usepackage{csquotes}

\usetheme{FSFW}
\date{18.~Dezember 2017}

\begin{document}

\title{Digitalisierung demokratisch gestalten}
\subtitle{Freie Software und Freie Inhalte für die Schule}

\maketitle

\begin{frame}
  \frametitle{Software in der Schule}

  \onslide<+->

  \begin{block}{Problemstellungen}
    Verwendung von Software in der Schule notwendig, \\
    aber nicht-technische Aspekte problematisch:
    \begin{itemize}
    \item Kosten
    \item Lizenzierung
    \item Datenschutz
    \item Herstellerneutralität
    \end{itemize}
  \end{block}
\end{frame}

\begin{frame}
  \frametitle{Software in der Schule (2)}

  \onslide<+->

  \begin{block}{Lösungsvorschlag}
    \emph{freie Software} statt proprietärer Programme
    \begin{itemize}
    \item manchmal nicht so \enquote{gut} wie kommerzielle Programme
    \item für den Schulgebrauch aber meist mehr als ausreichend!
    \end{itemize}
  \end{block}

  \onslide<+->

  \begin{block}{Achtung!}
    Aspekt \enquote{kostenlos} allein nicht ausreichend (mögliche
    Lizenzprobleme, mögliche Datenschutzprobleme, \emph{vendor
      lock-in})
  \end{block}

  \begin{block}{Probleme mit freien Alternativen}
    \begin{itemize}
    \item kein Allheilmittel
    \item Lehrkräfte haben meist nur indirekten Einfluss auf die
      IT-Infrastruktur
    \end{itemize}
  \end{block}
\end{frame}

\begin{frame}
  \frametitle{Beispiel~1: LibreOffice}

  \onslide<+->
  \begin{block}{Was?}
    \begin{itemize}
    \item freie Implementierung einer Office Suite (Textverarbeitung,
      Tabellenkalkulation, Präsentationen, \dots)
    \item verfügbar für Windows, Mac und Linux
    \end{itemize}
  \end{block}

  \onslide<+->
  \begin{block}{Warum?}
    \begin{itemize}
    \item für alle Schüler ohne Probleme verfügbar
    \item Unterstützung offener Dateiformate, die auch durch andere Programme
      lesbar sind
    \item keine Abhängigkeit von einem Hersteller
    \item keine Lizenzierung notwendig (insbesondere kostenlos)
    \end{itemize}
  \end{block}

  \onslide<+->

  \begin{block}{Woher?}
    \url{https://www.libreoffice.org/}
  \end{block}

  % TODO: screenshots from https://www.libreoffice.org/discover/screenshots/

\end{frame}

\begin{frame}
  \frametitle{Beispiel~2: OwnCloud/NextCloud}

  \onslide<+->

  \begin{block}{Was?}
    \begin{itemize}
    \item freies Programm zum Datenaustausch und zur Datensynchronisation
    \item verfügbar für Windows, Mac, Linux und Mobilgeräte
    \end{itemize}
  \end{block}

  \onslide<+->

  \begin{block}{Warum?}
    \begin{itemize}
    \item Kontrolle über eigene Daten
      \begin{itemize}
      \item Einsatz von Dropbox und Google Drive an Schulen nicht mit
        deutschem Datenschutzrecht vereinbar
      \end{itemize}
    \item keine Lizenzierung notwendig (insbesondere kostenlos)
    \end{itemize}
  \end{block}

  \onslide<+->

  \begin{block}{Woher?}
    \begin{itemize}
    \item \url{https://nextcloud.com/}
    \item \url{https://owncloud.org/}
    \end{itemize}
  \end{block}

  % TODO: screenshots

\end{frame}

\begin{frame}
  \frametitle{Beispiel~3: Gimp und Inkscape}

  \onslide<+->

  \begin{block}{Was?}
    \begin{itemize}
    \item Bildbearbeitung; pixelbasiert (Gimp) oder vektorbasiert (Inkscape)
    \item erhältlich für Windows, Mac und Linux
    \end{itemize}
  \end{block}

  \onslide<+->

  \begin{block}{Warum?}
    \begin{itemize}
    \item für Schulgebrauch mehr als ausreichend
    \item für alle Schüler ohne Probleme verfügbar
    \item keine Lizenzierung notwendig (insbesondere kostenlos)
    \end{itemize}
  \end{block}

  \onslide<+->

  \begin{block}{Woher?}
    \begin{itemize}
    \item \url{https://gimp.org}
    \item \url{https://inkscape.org}
    \end{itemize}
  \end{block}

  % TODO: screenshots

\end{frame}

\begin{frame}
  \frametitle{Weitere Beispiele}

  \onslide<+->

  \begin{itemize}
  \item GNU/Linux, ein freies Betriebssystem % eventuell skolelinux nennen?
                                % scheint aber nicht ganz aktuell zu sein :(
  \item GeoGebra (\url{https://www.geogebra.org})
  \item Stellarium (\url{www.stellarium.org})
  \item Minetest als Alternative zu Minecraft (\url{minetest.net})
  \item Jabber für Instant Messaging, zum Beispiel mit Pidgin
    (\url{https://pidgin.im})
  \item Scribus für Desktop Publishing (\url{https://www.scribus.net})
  \item ShotCut für Videobearbeitung (\url{https://www.shotcut.org})
  \item Audacity für Audiobearbeitung (\url{www.audacityteam.org})
  \item diaspora*, ein soziales Netzwerk (\url{https://www.joindiaspora.com})
  \item GNUSocial, Quitter, Mastodon, \dots{} Kurznachrichtendienste
    (\url{https://mastodon.social})
  \item Notepad++ (\url{https://notepad-plus-plus.org})
  \end{itemize}

  Weitere Beispiele unter \url{https://alternativeto.net}

\end{frame}

\begin{frame}
  \frametitle{Software-Freiheiten}

  \onslide<+->

  \begin{block}{Software für jede Anforderung benutzen zu können}
    \begin{itemize}
    \item Erlaubt die uneingeschränkte Verwendung von Programmen
    \item Schützt vor komplexen Nutzungsvereinbarungen und versehentlichen
      Lizenzverstößen; Verwendung wird erheblich vereinfacht
    \end{itemize}
  \end{block}

  \onslide<+->

  \begin{block}{Software untersuchen und modifizieren zu können}
    \begin{itemize}
    \item Erlaubt das Verständnis vorhandener Programme, die Anpassung an
      spezielle Anforderungen und die Überprüfung der Korrektheit
    \item Schützt vor Abhängigkeiten und unerwünschtem Verhalten
    \end{itemize}
  \end{block}

\end{frame}

\begin{frame}
  \frametitle{Software-Freiheiten}

  \onslide<+->

  \begin{block}{Kopien der Software weitergeben zu können}
    \begin{itemize}
    \item Erlaubt allen die Verwendung gleicher Programme ohne Berücksichtigung
      von sozialem oder finanziellem Status und fördert damit Zusammenarbeit
    \item Schützt vor sozialer Ausgrenzung
    \end{itemize}
  \end{block}

  \onslide<+->

  \begin{block}{Angepasste Kopien weitergeben zu können}
    \begin{itemize}
    \item Erlaubt allen Fehler zu beseitigen und die Korrekturen zu verbreiten
    \item Schützt vor Abhängigkeiten vom Hersteller
    \end{itemize}
  \end{block}
\end{frame}

\begin{frame}
  \frametitle{Weiterführendes zu Freier Software}

  \onslide<+->

  \begin{block}{Mehr Informationen zu Freier Software}
    \begin{itemize}
    \item Free Software Foundation (\url{https://fsf.org})
    \item Free Software Foundation Europe (\url{https://fsfe.org})
    \item Ausführliche Beschreibung der vier Freiheiten:
      \url{https://www.gnu.org/philosophy/free-sw.html}
    \end{itemize}
  \end{block}

\end{frame}


\begin{frame}
  \frametitle{Freie Inhalte für die Schule}

  \onslide<+->

  \begin{block}{Probleme mit Urheberrecht}
    \begin{itemize}
    \item Fragen zum Urheberrecht in Schule relevant (Arbeitsblätter,
      Aufgabensammlungen, \dots)
    \item Oft unklar, ob Kopien für Unterricht genutzt werden dürfen
      % http://www.urheberrecht.de/schule/
    \end{itemize}
  \end{block}

  \onslide<+->

  \begin{block}{Freie Inhalte}
    \begin{itemize}
    \item Freiheiten für Software auf Unterrichtsmaterialien übertragen
    \item Lizenzen, die Weitergabe und Veränderungen ermöglichen \\
      (Creative Commons)
    \item Ergebnis: \emph{Open Educational Resources}
    \item Beispiele
      \begin{itemize}
      \item Zentrale für Unterrichtsmedien im Internet
        (\url{https://www.zum.de})
      \item Serlo (\url{https://de.serlo.org})
      \item \url{https://www.tutory.de/}
      \item Wikipedia und Wikiversity
      \end{itemize}
    \end{itemize}
  \end{block}

\end{frame}

\begin{frame}
  \frametitle{Lernplattformen und freie Inhalte}
  \begin{block}{Probleme}
    \begin{itemize}
    \item werden oft als Schutz gegen Urheberrechtsansprüche verwendet
    \item für Dritte meist nicht zugänglich
    \item früher auf Webseiten für alle zugängliche Materialien
      verschwinden (z.\,B. Übungsaufgaben)
    \end{itemize}
  \end{block}

  \begin{block}{Beispiel OPAL}
    \begin{itemize}
    \item schlechte technische Umsetzung
    \item Probleme mit Barrierefreiheit
    \item Abhängigkeit vom Hersteller
    \item Inhalte nicht für Dritte zugänglich
    \end{itemize}
  \end{block}

\end{frame}

\begin{frame}
  \frametitle{Fazit}
  \begin{block}{Freie Software und freie Lerninhalte}
    \begin{itemize}
    \item unabhängig von profitorientierten Unternehmen
      (\enquote{Demokratie statt Oligarchie})
    \item vermeiden Lizenzprobleme
    \item schützen vor Abhängigkeiten und schwer zu kontrollierenden Kosten
      (Schulbuchmonopol)
    \item erlauben das schaffen kollektiver Werte und kooperative
      Verbesserung
    \item Alle können beitragen, alle profitieren!
    \end{itemize}
  \end{block}

\end{frame}

\begin{frame}
  \frametitle{Mehr von der FSFW}
  \onslide<+->

  % stolen from Datenspuren lightning talk

  % adapted by Carsten to make it more obvious that the keywords are URLs

  \begin{center}
  \parbox{7.3cm}{
    \href{https://fsfw-dresden.de/programm}{\texttt{{\color{gray!20}https://fsfw-dresden.de/}programm}}\\[1mm]
    \href{https://fsfw-dresden.de/mitmachen}{\texttt{{\color{gray!40}https://fsfw-dresden.de/}mitmachen}}\\[1mm]
    \href{https://fsfw-dresden.de/newsletter}{\texttt{{\color{gray!60}https://fsfw-dresden.de/}newsletter}}\\[1mm]
    \href{https://fsfw-dresden.de/sprechstunde}{\texttt{{\color{gray!60}https://fsfw-dresden.de/}sprechstunde}}\\[3mm]
  }\\[3mm]

    \includegraphics[width=50mm]{fsfw-netzwerke}\\[3mm]

    \parbox{7.8cm}{
    Fragen?
    \begin{itemize}
     \item[$\rightarrow$]\href{mailto:kontakt@fsfw-dresden.de}{\texttt{kontakt@fsfw-dresden.de}}
     \item[$\rightarrow$] (\LaTeX-)Sprechstunde (jeden 4. Dienstag im Monat, 19:00 Uhr, SLUB -2.115 )
    \end{itemize}

  }
  \end{center}

\end{frame}

\end{document}

% Local Variables:
% TeX-engine: luatex
% End:

%  LocalWords:  Nutzungsvereinbarungen
