\documentclass{beamer}

\usepackage{fontspec}
\usepackage[ngerman]{babel}
\usepackage{csquotes}

\usetheme{FSFW}

\begin{document}

\title{Digitale Selbstverteidigung}
\subtitle{Freie Software und Freie Inhalte für die Schule}

\maketitle

\begin{frame}
  \frametitle{Software in der Schule}

  \onslide<+->

  \begin{block}{Problem}
    Verwendung von Software in der Schule wünschenswert, aber
    nicht-technische Aspekte schwierig
    \begin{itemize}
    \item Kosten
    \item Lizenzierung
    \item Datenschutz
    \item Gleichberechtigung
    \end{itemize}
  \end{block}

  \onslide<+->

  \begin{block}{Achtung!}
    Aspekt \enquote{kostenlos} allein nicht ausreichend (mögliche
    Lizenzprobleme)
  \end{block}

  \onslide<+->

  \begin{block}{Alternativen}
    \emph{Freie Software} statt kommerzieller Programme
    \begin{itemize}
    \item meist nicht so \enquote{gut} wie kommerzielle Programme
    \item für den Schulgebrauch aber meist mehr als ausreichend!
    \end{itemize}
  \end{block}

\end{frame}

\begin{frame}
  \frametitle{Beispiel~1: LibreOffice}

  \onslide<+->

  \begin{block}{Was'n das?}
    \begin{itemize}
    \item freie Implementierung einer Office Suite (Textverarbeitung,
      Tabellenkalkulation, Präsentationen, \dots)
    \item verfügbar für Windows, Mac, und Linux
    \end{itemize}
  \end{block}

  \onslide<+->

  \begin{block}{Warum?}
    \begin{itemize}
    \item für alle Schüler ohne Probleme verfügbar
    \item Unterstützung offener Dateiformate, die auch durch andere Programme
      lesbar sind
    \item keine Abhängigkeit von einem Hersteller
    \item keine Lizenzierung notwendig (insbesondere kostenlos)
    \end{itemize}
  \end{block}

  \onslide<+->

  \begin{block}{Woher?}
    \url{https://www.libreoffice.org/}
  \end{block}

  % TODO: screenshots

\end{frame}

\begin{frame}
  \frametitle{Beispiel~2: OwnCloud/NextCloud}

  \onslide<+->

  \begin{block}{Was'n das?}
    \begin{itemize}
    \item Freies Programm zum Datenaustausch und zur Datensynchronisation
    \item verfügbar für Windows, Mac, Linux, und Mobilgeräte
    \end{itemize}
  \end{block}

  \onslide<+->

  \begin{block}{Warum?}
    \begin{itemize}
    \item Kontrolle über eigene Daten
    \item Einhaltung geltender Datenschutzregelungen (TODO: welcher?)
    \item keine Lizenzierung notwendig (insbesondere kostenlos)
    \end{itemize}
  \end{block}

  \onslide<+->

  \begin{block}{Woher?}
    \begin{itemize}
    \item \url{https://nextcloud.com/}
    \item \url{https://owncloud.org/}
    \end{itemize}
  \end{block}

  % TODO: screenshots

\end{frame}

\begin{frame}
  \frametitle{Beispiel~3: Gimp und Inkscape}

  \onslide<+->

  \begin{block}{Was'n das?}
    \begin{itemize}
    \item Bildbearbeitung; pixelbasiert (Gimp) oder vektorbasiert (Inkscape)
    \item erhältlich für Windows, Mac, und Linux
    \end{itemize}
  \end{block}

  \onslide<+->

  \begin{block}{Warum?}
    \begin{itemize}
    \item für Schulgebrauch mehr als ausreichend
    \item für alle Schüler ohne Probleme verfügbar
    \item keine Lizenzierung notwendig (insbesondere kostenlos)
    \end{itemize}
  \end{block}

  \onslide<+->

  \begin{block}{Woher?}
    \begin{itemize}
    \item \url{https://gimp.org}
    \item \url{https://inkscape.org}
    \end{itemize}
  \end{block}

  % TODO: screenshots

\end{frame}

\begin{frame}
  \frametitle{Weitere Beispiele}

  \onslide<+->

  \begin{itemize}
  \item GNU/Linux, ein freies Betriebssystem % eventuell skolelinux nennen?
                                % scheint aber nicht ganz aktuell zu sein :(
  \item GeoGebra (\url{https://www.geogebra.org/})
  \item Stellarium (\url{www.stellarium.org})
  \item Minetest, eine freie Alternative zu Minecraft (\url{minetest.net})
  \item Jabber/XMPP, ein freies und dezentrales Instant Messaging Protokoll
  \item TODO: welche noch? % diaspora*? Mastodon?  What else?
  \end{itemize}

\end{frame}

\begin{frame}
  \frametitle{Software-Freiheiten}

  % TODO: Noch mehr auf Inhalte von
  % https://www.gnu.org/philosophy/free-sw.en.html
  % eingehen

  \onslide<+->

  \begin{enumerate}

  \item<+-> Software für jeden Anforderung benutzen zu können
    \begin{itemize}
    \item Erlaubt die uneingeschränkte Verwendung von Programmen
    \item Schützt vor komplexen Nutzungsvereinbarungen und versehentlichen
      Lizenzverstößen; Verwendung wird erheblich vereinfacht.
    \end{itemize}

  \item<+-> Software untersuchen und modifizieren zu können
    \begin{itemize}
    \item Erlaubt das Verständnis vorhandener Programme, die Anpassung an
      spezielle Anforderungen, und die Überprüfung der Korrektheit
    \item Schützt vor Abhängigkeiten und unerwünschtem Verhalten
    \end{itemize}

  \item<+-> Kopien der Software weitergeben zu können
    \begin{itemize}
    \item Erlaubt allen die Verwendung gleicher Programme, ohne Berücksichtigung
      von sozialem oder finanziellem Status, und fördert damit Zusammenarbeit
    \item Schützt vor sozialer Ausgrenzung
    \end{itemize}

  \item<+-> Angepasste Kopien weitergeben zu können
    \begin{itemize}
    \item Erlaubt allen Beteiligten, von Anpassungen anderer zu profitieren
    \item Schützt vor Abhängigkeiten bei Fehlerkorrekturen und vor unerwünschtem
      Verhalten der Software
    \end{itemize}
  \end{enumerate}

\end{frame}

\begin{frame}
  \frametitle{Freie Inhalte für die Schule}

  % Freie Inhalte für die Schule und weshalb diese sinnvoll sind

\end{frame}

\end{document}

% Local Variables:
% TeX-engine: luatex
% End:

%  LocalWords:  Nutzungsvereinbarungen
